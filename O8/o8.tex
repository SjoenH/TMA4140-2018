\documentclass[12pt]{article}
    \usepackage{mathtools}
    \usepackage[hidelinks]{hyperref}
    \usepackage{color}
    \usepackage{fancyref}
    \usepackage{lastpage}
    \usepackage{fancyhdr}
    
    \usepackage{pagecolor}
    % \pagecolor{black}
    % \color{white}

    \pagestyle{fancy}
    \fancyhf{}
    \fancyhead[LO]{TMA4140: Homework set 8}
    \fancyhead[RO]{Henry S. Sjøen \& Toralf Tokheim}
    \fancyfoot[CO]{\thepage\ of \pageref{LastPage}}

    \definecolor{darkred}{RGB}{200, 0, 0}

  \author{Henry S. Sjøen \& Toralf Tokheim}
  \title{
  \textbf{TMA4140 - Homework Exercise Set 8}}
\begin{document}
    \maketitle
    \thispagestyle{empty}
    \pagebreak
    \tableofcontents
    \pagebreak

    \section{Section 5.2}
    %  Obligatoriske: 4, 14; Anbefalte: 7, 19, 23
    \subsection{Todo: Exercise 4}
    Let $P(n)$ be the statement that a postage of $n$ cents can be formed using just 4-cent stamps and 7-cent stamps. The parts of this exercise outline a strong induction proof that $P(n)$ is $true$ for $n\geq 18$.
    \textbf{a)} Show statements $P(18),P(19),P(20)$ and $P(21)$ are $true$, completing the basis step of the proof.
    \textbf{b)} What is the inductive hypothesis of the proof?
    \textbf{c)} What do you need to prove in the inductive step?
    \textbf{d)} Complete the inductive step for $k \geq 21$.
    \textbf{e)} Explain why these steps show that this statement is $true$ whenever $n\geq 18$.

    \subsection{Todo: Exercise 14}
    Suppose you begin with a pile of $n$ stones and split this pile into $n$ piles of one stone each by successively splitting a pile of stones into two smaller piles. Each time you split a pile you multiply the number of stones in each of the two smaller piles you form, so that if these piles have $r$ and $s$ stones in them, respectively, you compute $rs$. Show that no matter how you split the piles, the sum of the products computed at each step equals $n(n-1)/2$.

    \section{Section 5.3}
    %  Obligatoriske: 12, 18; Anbefalte: 13, 14, 15
    \subsection{Todo: Exercise 12}
    Prove that $f_1^2 +f_2^2 +\cdots+f_n^2 =f_nf_{n+1}$ when $n$ is a positive integer.

    \subsection{Todo: Exercise 18}
    Let 
    $ A =
    \begin{bmatrix}
        1 & 1 \\
        1 & 0  
    \end{bmatrix}
    $,
    Show that 
    $ A^n = \begin{bmatrix}
                    f_{n+1} & f_n\\
                    f_n & f_{n-1}
                    \end{bmatrix}
    $   when $n$ is a positive integer.

    \section{Section 5.4}
    %  Obligatoriske: 3
    \subsection{Todo: Exercise 3}
    Trace Algorithm 3 when it finds $gcd(8,13)$. That is, show all the steps used by Algorithm 3 to find $gcd(8,13)$.

    \section{Section 9.1}
    %  7, 40a, 40c (4, 28a, 28c); Anbefalte: 4, 25, 32
    \subsection{Todo: Exercise 7}
    Determine whether the relation $R$ on the set of all integers is \textit{reflexive}, \textit{symmetric}, \textit{antisymmetric}, and/or \textit{transitive}, where $(x,y) \in R$ if and only if\\
    \textbf{a)} $ x \neq y$.\\
    \textbf{b)} $ xy \geq 1$.\\
    \textbf{c)} $ x=y + 1$ or $ x=y-1$.\\
    \textbf{d)} $ x \equiv y (mod 7) $.\\
    \textbf{e)} $ x $ is a multiple of $y$.\\
    \textbf{f)} $ x $ and $y$ are both negative or both nonnegative.\\
    \textbf{g)} $ x=y^2 $.\\
    \textbf{h)} $ x \geq y^2 $.

    \subsection{Todo: Exercise 40}
    Let $R_1$ and $R_2$ be the "divides" and "is a multiple of" relations on the set of all positive integers, respectively.\\
    That is, $R_1 = \{(a,b)|a $ divides $ b\}$ and $R_2=\{(a,b)|a$ is a multiple of $b\}$. Find...
    
    \subsubsection{Todo: Exercise 40.a}
    $R_1 \cup R_2$.
    
    \subsubsection{Todo: Exercise 40.c}
    $R_1 - R_2$.

    \section{Section 9.3}
    %  10, 14a, 14b, 14c (6, 10a, 10b, 10c); Anbefalte: 15
    \subsection{Todo: Exercise 10}
    How many nonzero entries does the matrix representing the relation $R$ on $A=\{1,2,3,...,1000\}$ consisting of the first 1000 positive integers have if $R$ is...\\
    \textbf{a)} $\{(a,b) | A \leq b \}$?\\
    \textbf{b)} $\{(a,b) | a=b \pm 1 \}$?\\
    \textbf{c)} $\{(a,b) | a+b = 1000 \}$?\\
    \textbf{d)} $\{(a,b) | a+b \leq 1001 \}$?\\
    \textbf{e)} $\{(a,b) | a \neq 0 \}$?
    \subsection{Todo: Exercise 14}
    Let $R_1$ and $R_2$ be the relations represented by the matrices... \\
    $M_{R_1}=\begin{bmatrix}
        0 & 1 & 0\\
        1 & 1 & 1\\
        1 & 0 & 0
    \end{bmatrix}$ and $M_{R_2} = \begin{bmatrix}
        0 & 1 & 0 \\
        0 & 1 & 1 \\
        1 & 1 & 1
    \end{bmatrix}
    $\\
    Find the matrices that represent...
    \subsubsection{Todo: Exercise 14.a}
    $R_1 \cup R_2$
    \begin{equation}
        R_1 \cup R_2 = \begin{bmatrix}
            0 & 1 & 0 \\
            1 & 1 & 1 \\
            1 & 1 & 1
        \end{bmatrix}
    \end{equation}

    \subsubsection{Todo: Exercise 14.b}
    $ R_1 \cap R_2 $.
    \begin{equation}
        R_1 \cap R_2=
        \begin{bmatrix}
            0 & 1 & 0\\
            0 & 1 & 1\\
            1 & 0 & 0
        \end{bmatrix}
    \end{equation}
    
    \subsubsection{Todo: Exercise 14.c}
    $ R_2 \circ R_1$.
    \begin{equation}
        R_2 \circ R_1
    \end{equation}
\end{document}