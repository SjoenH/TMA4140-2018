\documentclass[12pt]{article}
    \usepackage{mathtools}
    \usepackage[hidelinks]{hyperref}
    \usepackage{color}
    \usepackage{fancyref}
    \usepackage{lastpage}
    \usepackage{fancyhdr}
    
    \usepackage{pagecolor}
    % \pagecolor{black}
    % \color{white}

    \pagestyle{fancy}
    \fancyhf{}
    \fancyhead[LO]{TMA4140: Homework set 8}
    \fancyhead[RO]{Henry S. Sjøen \& Toralf Tokheim}
    \fancyfoot[CO]{\thepage\ of \pageref{LastPage}}

    \definecolor{darkred}{RGB}{200, 0, 0}

  \author{Henry S. Sjøen \& Toralf Tokheim}
  \title{
  \textbf{TMA4140 - Homework Exercise Set 8}}
\begin{document}
    \maketitle
    \thispagestyle{empty}
    \pagebreak
    \tableofcontents
    \pagebreak

    \section{Section 5.2}
    %  Obligatoriske: 4, 14; Anbefalte: 7, 19, 23
    \subsection{Exercise 4}
    Let $P(n)$ be the statement that a postage of $n$ cents can be formed using just 4-cent stamps and 7-cent stamps. The parts of this exercise outline a strong induction proof that $P(n)$ is $true$ for $n\geq 18$.\\
    
    \textbf{a)} Show statements $P(18),P(19),P(20)$ and $P(21)$ are $true$, completing the basis step of the proof.
    
    \begin{equation}
        \begin{split}
            P(n)=4a+7b=n\\
            P(18)=(4\times 1)+(7\times 2)=18\\
            P(19)=(4\times 3)+(7\times 1)=19\\
            P(20)=(4\times 5)+(7\times 0)=20\\
            P(21)=(4\times 0)+(7\times 3)=21\\
        \end{split}
    \end{equation}
    
    \textbf{b)} What is the inductive hypothesis of the proof?\\
    If for every i in $18 \leq i \leq k $, where $k \leq 21$, there is an "a" and "b" so that $i= 4a+7b$. Then there is an  "c" and "b" so that $k+1 = 4c+7d$ is \textit{true}.\\

    \textbf{c)} What do you need to prove in the inductive step?\\
    In the inductive step, we assume that the inductive hypothesis holds, and use that to prove $k+1$. \\
    
    \textbf{d)} Complete the inductive step for $k \geq 21$.\\
    For $k = 21$ then $P(k+1)$ should still be \textit{true}.
    \begin{equation}
        \begin{split}
                    P(k)=(4\times 0)+(7\times 3)=21\\
                    P(k+1)=4+(k-3)=(4\times 2)+(7\times 2)=22\\
        \end{split}
    \end{equation}
    \textit{True} for base step $P(21)$ and the first inductive step $P(22)$.\\
    
    \textbf{e)} Explain why these steps show that this statement is $true$ whenever $n\geq 18$.\\
    Since the base step and the inductive step are true, by the principle of strong induction all amount of postage where $n \geq 18$ can be obtained using only 4- and 7-cent stamps.

    % \subsection{Todo: Exercise 14}
    % Suppose you begin with a pile of $n$ stones and split this pile into $n$ piles of one stone each by successively splitting a pile of stones into two smaller piles.\\
    % Each time you split a pile you multiply the number of stones in each of the two smaller piles you form, so that if these piles have $r$ and $s$ stones in them, respectively, you compute $rs$. Show that no matter how you split the piles, the sum of the products computed at each step equals $n(n-1)/2$.

    \section{Section 5.3}
    %  Obligatoriske: 12, 18; Anbefalte: 13, 14, 15
    \subsection{Exercise 12}
    Prove that $f_1^2 +f_2^2 +\cdots+f_n^2 =f_nf_{n+1}$ when $n$ is a positive integer.
    \begin{equation}
        \begin{split}
                    f_{0} = 0\\
                    f_{1} = 1\\
                    \\
                    f_{2} = f_{1}+f_{0}=1+0=1 \\
                    f_{3} = f_{2}+f_{1}=1+1=2 \\
                    f_{4} = f_{3}+f_{1}=2+1=3 \\
        \end{split}
    \end{equation}
    
    \textbf{Base step:} \\
    $P(1)$ is true because $f_{1}^2 = f_{1}f_{1}$\\
    $= 1.1$\\
    $= 1$\\
    $= f_1*f_2$\\
    \\
    
    \textbf{Inductive step:}\\
    Assume that $P(k)$  is true.\\
    Show that $P(k+1)$ is true:
    \begin{equation}
        \begin{split}
              f_{1}^2 + f_{2}^2 + \cdots + f_k^2 + f_{k+1}^2 = f_k * f_{k+1} + f_{k+1}^2\\
            = f_k * f_{k+1} + f_{k+1} * f_{k+1}\\
            = f_{k+1} (f_k+f_{k+1})
            =f_{k+1} * f_{k+2}
        \end{split}
    \end{equation}
    Therefore, $P(k+1)$ is true. \\
    Hence by the principle of strong induction, we conclude that the statement is true. \\
    And therefor $f_1^2 +f_2^2 +\cdots+f_n^2 =f_nf_{n+1}$ is true.

    \subsection{Exercise 18}
    Let 
    $ A =
    \begin{bmatrix}
        1 & 1 \\
        1 & 0  
    \end{bmatrix}
    $,
    Show that 
    $ A^n = \begin{bmatrix}
                    f_{n+1} & f_n\\
                    f_n & f_{n-1}
                    \end{bmatrix}
    $   when $n$ is a positive integer.
    
    Let $A=\begin{bmatrix}
    1&1\\
    1&0
    \end{bmatrix}$
    Let $P(n)$ be a statement that $A^n=\begin{bmatrix}
        f_{n+1}&f_{n}\\
        f_{1}&f_{1-1}
    \end{bmatrix}$
    
    \begin{equation}
        A^1= \begin{bmatrix}
            f_{2}&f_{1}\\
            f_{1}&f_{0}
        \end{bmatrix}\\
        =\begin{bmatrix}
            1&1\\
            1&0
        \end{bmatrix}
        = A
    \end{equation}
    A is true.\\
    Inductive step...
    Assume that $P(k)$ is true.
    i.e., $A^k=\begin{bmatrix}
    f_{k+1}&f_k\\
    f_k & f_k-1
    \end{bmatrix}$\\
    We have to prove that $P(k+1)$ is true.
    \begin{equation}
        \begin{split}
            A^{k+1}&=A.A^k\\
            &=\begin{bmatrix}
            1&1\\
            1&0
            \end{bmatrix}
            \begin{bmatrix}
            f_{k+1}& f_k\\
            f_k&f_{k-1}
            \end{bmatrix}\\
            &=\begin{bmatrix}
            1.f_{k+1}+1.f_k &1.f_{k}+1.f_{k-1}\\
            1.f_{k+1}+ 0.f_k & 1.f_k+0.f_{k-1}
            \end{bmatrix}\\
            &=\begin{bmatrix}
            f_{k+1}+f_k & f_{k-1}+f_k\\
            f_{k+1}+0 & f_{k}+0
            \end{bmatrix}\\
            &=\begin{bmatrix}
            f_{k+2} & f_{k+1}\\
            f_{k+1} & f_{k}
            \end{bmatrix}
        \end{split}
    \end{equation}
    Therefore $P(k+1)$ is \textbf{true}.\\
    And from the principle of mathematical induction we can conclude that the given statement is true.
    
    \section{Section 5.4}
    %  Obligatoriske: 3
    \subsection{Exercise 3}
    Trace Algorithm 3 when it finds $\gcd(8,13)$. That is, show all the steps used by Algorithm 3 to find $\gcd(8,13)$.
    
            \textbf{Input:} $\gcd(8, 13)$ \\
            \begin{equation}
                \begin{split}
                   since(8 < 13) &\\
                        \gcd(8, 13) &= \gcd(13 \textbf{mod} 8,8)\\
                        \gcd(8, 13) &= \gcd(5, 8) \\
                        \\
                   since(5 < 8) &\\
                        \gcd(5, 8) &= \gcd(8 \textbf{mod} 5,5)\\
                        \gcd(5, 8) &= \gcd(3, 5) \\
                        \\
                   since(3 < 5) &\\
                        \gcd(3, 5) &= \gcd(5 \textbf{mod} 3,3)\\
                        \gcd(3, 5) &= \gcd(2, 3) \\
                        \\
                   since(2 < 3) &\\
                        \gcd(2, 3) &= \gcd(3 \textbf{mod} 2,2)\\
                        \gcd(2, 3) &= \gcd(1, 2) \\
                        \\
                   since(1 < 2) &\\
                        \gcd(1, 2) &= \gcd(2 \textbf{mod} 1,1)\\
                        \gcd(1, 2) &= \gcd(0, 1) \\
                        \\
                   since(a = 0) &\\
                        \gcd(0, 1) &= 1\\
                        \gcd(8, 13) &= \textbf{1}
                \end{split}
            \end{equation}


    \section{Section 9.1}
    %  7, 40a, 40c (4, 28a, 28c); Anbefalte: 4, 25, 32
    \subsection{Exercise 7}
    Determine whether the relation $R$ on the set of all integers is \textit{reflexive}, \textit{symmetric}, \textit{antisymmetric}, and/or \textit{transitive}, where $(x,y) \in R$ if and only if\\
    \textbf{a)} $ x \neq y$.\\
        $R$ is \textbf{Not reflective}. $x \neq x$ can never be true\\
        $R$ is \textbf{symetric}, since if $x, y$ are integers and $ x \neq y$; then $ y \neq x$. \\
        $R$ is \textbf{not antisymetric}, as $1\neq 5$, $5 \neq 1$ while $1$ and $5$ are not the same. 
        $R$ is \textbf{not transitive}, as $1\neq 5$, $5 \neq 1$ while $1 = 1$\\
        \\
    \textbf{b)} $ xy \geq 1$.\\
        $R$ is \textbf{not reflexive} \\
        $R$ is \textbf{symetric} \\
        $R$ is \textbf{not antisymetric} \\
        $R$ is \textbf{transitive} \\
        \\
    \textbf{c)} $ x=y + 1$ or $ x=y-1$.\\
        $R$ is \textbf{not reflexive} \\
        $R$ is \textbf{symetric} \\
        $R$ is \textbf{not asymetric} \\
        $R$ is \textbf{not transitive} \\
        \\
    \textbf{d)} $ x \equiv y (mod 7) $.\\
        $R$ is \textbf{reflexive} \\
        $R$ is \textbf{symetric} \\
        $R$ is \textbf{not asymetric} \\
        $R$ is \textbf{transitive} \\
        \\
    \textbf{e)} $ x $ is a multiple of $y$.\\
        $R$ is \textbf{reflective} \\
        $R$ is \textbf{not symmetric} \\
        $R$ is \textbf{not antisymetric} \\
        $R$ is \textbf{transitive} \\
        \\
    \textbf{f)} $ x $ and $y$ are both negative or both nonnegative.\\
        $R$ is \textbf{reflexitive} \\
        $R$ is \textbf{symmetric} \\
        $R$ is \textbf{not asymmetric} \\
        $R$ is \textbf{transitive} \\
        \\
    \textbf{g)} $ x=y^2 $.\\
        $R$ is \textbf{not reflexive} \\
        $R$ is \textbf{not symmetric} \\
        $R$ is \textbf{asymmetric} \\
        $R$ is \textbf{not transitive} \\
        \\
    \textbf{h)} $ x \geq y^2 $.
        $R$ is \textbf{not reflexive} \\
        $R$ is \textbf{not symmetric} \\
        $R$ is \textbf{asymmetric} \\
        $R$ is \textbf{transitive} \\
        \\

    \subsection{Exercise 40}
    Let $R_1$ and $R_2$ be the "divides" and "is a multiple of" relations on the set of all positive integers, respectively.\\
    That is, $R_1 = \{(a,b)|a $ divides $ b\}$ and $R_2=\{(a,b)|a$ is a multiple of $b\}$. Find...
    
    \subsubsection{Exercise 40.a}
    $R_1 \cup R_2$.\\
    An ordered pair $(a,b)\in R_1\cup R_2 $ if and only if $(a,b)\in R_1$ or $(a,b)\in R_2$.\\
    If $a$ divides $b$ for some integer $k$, $b=ka$.\\
    If $a$ is an multiple of $b$, then for some integer k,
    $b=\frac{1}{k}a $\\
    Therefore, $R_1 \cup R_2=\{(a,b)|b=ka$ or $ b=\frac{1}{k}a$, where $k$ is an integer.$\}$
    
    \subsubsection{Exercise 40.c}
    $R_1 - R_2$.\\
    An ordered pair $(a,b)\in R_1-R_2$ if and only if $(a,b)\in R_1$ and $(a,b)\notin R_2$.\\
    That is, if and only if \textit{a divides b} and \textit{a} is not a multiple of \textit{b}.\\
    Therefore,
    $R_1-R_2 = \{(a,b) | a$ is a proper divisor of $b \}$.

    \section{Section 9.3}
    %  10, 14a, 14b, 14c (6, 10a, 10b, 10c); Anbefalte: 15
    \subsection{Exercise 10}
    How many nonzero entries does the matrix representing the relation $R$ on $A=\{1,2,3,...,1000\}$ consisting of the first 1000 positive integers have if $R$ is...\\
    \textbf{a)} $\{(a,b) | A \leq b \}$?
    \begin{equation}
        \begin{split}
            \{(a,b) | A \leq b \}&=1000+999+998+\cdots + 2 + 1\\
            &=\frac{(1000)(1000+1)}{2}\\
            &=(500)(1001)\\
            &=500,500
        \end{split}
    \end{equation}
    
    \textbf{b)} $\{(a,b) | a=b \pm 1 \}$?
    \begin{equation}
        \begin{split}
            \{(a,b) | a=b \pm 1 \}&=1+2(998)+1\\
            &=1+1996+1\\
            &=1998
        \end{split}
    \end{equation}
    
    \textbf{c)} $\{(a,b) | a+b = 1000 \}$?
    \begin{equation}
        \begin{split}
            \{(a,b) | a+b = 1000 \}&=1+1+1+\cdots + 1(999)\\
            &=999
        \end{split}
    \end{equation}
    
    \textbf{d)} $\{(a,b) | a+b \leq 1001 \}$?
        \begin{equation}
            \begin{split}
            \{(a,b) | a+b \leq 1001 \} &=1000+999+998+\cdots+2+1\\
            &=\frac{(1000)(1000+1)}{2}\\
            &=(500)(1001)\\
            &=500,500
            \end{split}
        \end{equation}
   
    \textbf{e)} $\{(a,b) | a \neq 0 \}$?
        \begin{equation}
            \begin{split}
            \{(a,b) | a \neq 0 \}&=1000+1000+1000+\cdots + 1000+1000 (1000 times)\\
            &=(1000)(1000)\\
            &= 1,000,000
            \end{split}
    \end{equation}
     
    \subsection{Exercise 14}
    
    Let $R_1$ and $R_2$ be the relations represented by the matrices... \\
    $M_{R_1}=\begin{bmatrix}
        0 & 1 & 0\\
        1 & 1 & 1\\
        1 & 0 & 0
    \end{bmatrix}$ and $M_{R_2} = \begin{bmatrix}
        0 & 1 & 0 \\
        0 & 1 & 1 \\
        1 & 1 & 1
    \end{bmatrix}
    $\\
    Find the matrices that represent...
    \subsubsection{Exercise 14.a}
    $R_1 \cup R_2$
    \begin{equation}
        M_{R_1 \cup R_2} = \begin{bmatrix}
            0 & 1 & 0 \\
            1 & 1 & 1 \\
            1 & 1 & 1
        \end{bmatrix}
    \end{equation}

    \subsubsection{Exercise 14.b}
    $ R_1 \cap R_2 $.
    \begin{equation}
        M_{ R_1 \cap R_2}=
        \begin{bmatrix}
            0 & 1 & 0\\
            0 & 1 & 1\\
            1 & 0 & 0
        \end{bmatrix}
    \end{equation}
    
    \subsubsection{Exercise 14.c}
    $ R_2 \circ R_1$.
    \begin{equation}
       M_{R_2 \circ R_1}=
        \begin{bmatrix}
        0&1&1\\
        1&1&1\\
        0&1&0
        \end{bmatrix}
    \end{equation}
\end{document}