\documentclass[12pt]{article}
    \usepackage{mathtools}
    \usepackage[hidelinks]{hyperref}
    \usepackage{color}
    \usepackage{fancyref}
    \usepackage{lastpage}
    \usepackage{fancyhdr}
    
    \usepackage{pagecolor}
    % \pagecolor{black}
    % \color{white}

    \pagestyle{fancy}
    \fancyhf{}
    \fancyhead[LO]{TMA4140: Homework set 7}
    \fancyhead[RO]{Henry S. Sjøen \& Toralf Tokheim}
    \fancyfoot[CO]{\thepage\ of \pageref{LastPage}}

    \definecolor{darkred}{RGB}{200, 0, 0}

  \author{Henry S. Sjøen \& Toralf Tokheim}
  \title{
  \textbf{TMA4140 - Homework Exercise Set 7}}
\begin{document}
    \maketitle
    \thispagestyle{empty}
    \pagebreak
    \tableofcontents
    \pagebreak

    \section{Section 8.1} 
    % Obligatoriske: 11, 20; 
    % Anbefalte: 19, 21.
    \subsection{Exercise 11}
    \textbf{a) Find a recurrence relation for the number of ways to climb n stairs if the person climbing the stairs can take one stair or two stairs at a time.}\\
    $ a_n=a_{n-1}+a_{n-2} $ when $n > 2$\\
    \textbf{b) What are the initial conditions?}\\
    % Only one way to reach "The zero-step", is not to move, witch also is a move. $a_0 = 1$\\
    To reach the first step there is only one solution, taking one single step $a_1=1$\\
    To reach the second step, we can take either two single steps or one double step. $a_2=2$\\
    \textbf{c) In how many ways can this person climb a flight of eight stairs?}\\
    A lot... 
    \begin{equation}
        \begin{split}
            % 8 enere
            % 0 toer
            1+1+1+1+1+1+1+1=8\\
            % 6 enere
            % 1 toer
            % Gir oss 7 bokser
            % og også 7 permutasjoner
            2+1+1+1+1+1+1=8\\
            1+2+1+1+1+1+1=8\\
            1+1+2+1+1+1+1=8\\
            1+1+1+2+1+1+1=8\\
            1+1+1+1+2+1+1=8\\
            1+1+1+1+1+2+1=8\\
            1+1+1+1+1+1+2=8\\
            % 4 enere
            % 2 toere
            % Gir oss 6 bokser
            2+2+1+1+1+1=8\\
            2+1+2+1+1+1=8\\
            2+1+1+2+1+1=8\\
            2+1+1+1+2+1=8\\
            2+1+1+1+1+2=8\\
            % 
            1+2+2+1+1+1=8\\
            1+2+1+2+1+1=8\\
            1+2+1+1+2+1=8\\
            1+2+1+1+1+2=8\\
            % 
            % 
            1+1+2+2+1+1=8\\
            1+1+2+1+2+1=8\\
            1+1+2+1+1+2=8\\
            % 
            1+1+1+2+2+1=8\\
            1+1+1+2+1+2=8\\
            % 
            1+1+1+1+2+2=8\\
            % 
            % 2 enere
            % 3 toere
            2+2+2+1+1=8\\
            2+2+1+2+1=8\\
            2+2+1+1+2=8\\
            2+1+2+1+2=8\\
            2+1+1+2+2=8\\
            1+2+2+2+1=8\\
            1+2+2+1+2=8\\
            1+2+1+2+2=8\\
            1+1+2+2+2=8\\
            % 0 enere
            % 4 toere
            2+2+2+2=8\\
        \end{split}
    \end{equation}

    I missed one above, but easier to calculate it...
    $ a_n=a_{n-1}+a_{n-2} $ when $n > 2$\\

    \begin{equation}
        \begin{split}
            a_n&=a_{n-1}+a_{n-2} | n>2\\
            % a_0&=1,
            a_1&=1,a_2=2\\ 
            % a_2&=a_{2-1}+a_{2-2}=a_{1}+a_{0}=1+1=2\\
            a_3&=a_{3-1}+a_{3-2}=a_{2}+a_{1}=2+1=3\\
            a_4&=a_{4-1}+a_{4-2}=a_{3}+a_{2}=3+2=5\\
            a_5&=a_{5-1}+a_{5-2}=a_{4}+a_{3}=5+3=8\\
            a_6&=a_{6-1}+a_{6-2}=a_{5}+a_{4}=8+5=13\\
            a_7&=a_{7-1}+a_{7-2}=a_{6}+a_{5}=13+8=21\\
            a_8&=a_{8-1}+a_{8-2}=a_{7}+a_{6}=21+13=34
        \end{split}
    \end{equation}

    There are 34 possible ways to climb the flight of $8$ stairs using only $1$ or $2$ steps at a time.

    \subsection{Exercise 20}
    % https://www.chegg.com/homework-help/bus-driver-pays-tolls-using-nickels-dimes-throwing-one-coin-chapter-8.1-problem-20e-solution-9780073383095-exc
    A bus driver pays all tolls, using only nickels and dimes, by throwing one coin at a time into the mechanical toll collector.\\
    \textbf{a)} Find a recurrence relation for the number of different ways the bus driver can pay a toll of n cents (where the order in which the coins are used matters).\\
    1 Nickel =  5 cents, 1 Dime = 10 cents.\\
    Let $a_n$ be the number of ways the busdriver can pay a toll of $n$ cents.\\
    If last coin Nickel $a_{n-5}$, If last coin Dime $a_{n-10}$\\
    Then the recurrence relation can be given as...
    $a_n=a_{n-5}+a_{n-10} | n \geq 10$
    And since nickels and dimes are of multiple of 5 we can further write it as ...
    $a_{5n}=a_{5(n-1)} + a_{5(n-2)} | n \geq 2$.
    Where the initial conditions are $a_0=1; a_5=1;$
    \textbf{b)} In how many different ways can the driver pay a toll of 45 cents?
    We must calculate $a_{45}$
    \begin{equation}
        \begin{split}
            a_{5n}&=a_{5(n-1)} + a_{5(n-2)}\\
            a_0&=1; a_5=1\\
            a_{10}&=2\\
            a_{15}&=3\\
            a_{20}&=5\\
            a_{25}&=8\\
            a_{30}&=13\\
            a_{35}&=21\\
            a_{40}&=34\\
            a_{45}&=55
        \end{split}
    \end{equation}

    \section{Section 8.2} 
    % Obligatoriske: 3c,d,e,g, 6, 11, 42; 
    % Anbefalte: 40.
    
    \subsection{TODO: Exercise 3}
    % https://www.chegg.com/homework-help/bus-driver-pays-tolls-using-nickels-dimes-throwing-one-coin-chapter-8.2-problem-3E-solution-9780073383095-exc
    Solve these recurrence relations together with the initial conditions given.\\
    \textbf{c)} $ a_n=5a_{n-1} - 6a_{n-2} $ for $ n \geq 2, a_0=1, a_1=0 $\\
    Comparing the given recurrence with the general relation we get $C_1=5,c_2=-6$ and rest of the coefficients are $0$.
    Our characteristic equation will be $r^2=5r^1-6r^0$, so...
    \begin{equation}
        \begin{split}
            r^2-5r^1+6r^0=0\\
            r^2-5r+6r=0\\
            r^2-3r-2r+6=0\\
            (r-3)(r-2)=0
        \end{split}
    \end{equation}
    That is, $r=3,2$.
    Solution will be of the form $a_n=a_1r_2^n$. Setting in the value of $r$ and use the initial condition.
    $n=0,1$...\\
    \begin{equation}
        \begin{split}
            a_n=a_1r_1^n\\
            a_n=a_1(2)^n+a_2(3)^n
        \end{split}
    \end{equation}
    When $n=0$, then...
    \begin{equation}
        \begin{split}
            a_0=a_1(2)^0+a_2(3)^0\\
            1=a_1+a_2\\
            a_1=1-a_2
        \end{split}
    \end{equation}
    When $n=1$, then...
    \begin{equation}
        \begin{split}
            a_1=a_1(2)^1+a_2(3)^1\\
            0=2a_1+3a_2
        \end{split}
    \end{equation}
    Put the value of $a_1=1-a_2$ in the equation $2a_1+3a_2=0$...
    \begin{equation}
        \begin{split}
            2(1-a_2)+3a_2=0\\
            2-2a_2+3a_2=0\\
            a_2=-2
        \end{split}
    \end{equation}
    Put the value of $a_2=-2$ in the equation $a_1=1-a_2$ and we get $a_1=3$.
    Put the value of $a_1,a_2$ in the equation $a_n=a_1r_1^n+a_2r_2^n$ to get the final solution $a_n=3 \times (2)^n-2 \times (3)^n$.

    \textbf{d)} $ a_n=4a_{n-1} - 4a_{n-2} $ for $ n \geq 2, a_0=6, a_1=8 $\\
    Comparing the given recurrence relation with the general relation we get $c_1=4$ $c_2=-4$ and the rest of the coefficients are 0.
    The corresponding characteristic equation will be $r^2=4r^1-4r^0$. So,
    \begin{equation}
        \begin{split}
            r^2-4r+4=0\\
            r^2-2r-2r+4=0\\
            (r-2)(r-2)=0
        \end{split}
    \end{equation}
    Since we have a repeated root, the solution will be of the form $a_n=a_1r_1^n+a_2nr_2^n$
    \textbf{e)} $ a_n=-4a_{n-1} - 4a_{n-2} $ for $ n \geq 2, a_0=0, a_1=1 $\\
    \textbf{g)} $ a_n=\frac{a_{n-2}}{4} $ for $ n \geq 2, a_0=1, a_1=0 $
    \subsection{Todo: Exercise 6}
    % https://www.chegg.com/homework-help/bus-driver-pays-tolls-using-nickels-dimes-throwing-one-coin-chapter-8.2-problem-6E-solution-9780073383095-exc
    How many different messages can be transmitted in $n$ microseconds using three different signals if one signal requires $1$ microsecond for transmittal, the other two signals require $2$ microseconds each for transmittal, and a signal in a message is followed immediately by the next signal?

    \subsection{Todo: Exercise 11}
    % https://www.chegg.com/homework-help/bus-driver-pays-tolls-using-nickels-dimes-throwing-one-coin-chapter-8.2-problem-11E-solution-9780073383095-exc
    The Lucas numbers satisfy the recurrence relation
    \begin{equation}
        L_n=L_{n-1}+L_{n-2}
    \end{equation}
    and the initial conditions $L_0 = 2$ and $L_1=1$.

    \textbf{a)} Show that $L_n = f_{N-1}+f_{n+1}$ for $n=2,3,...,$ where $f_n$ is the $n$th Fibonacci number.\\
    \textbf{b)} Find an explicit formula for the Lucas numbers.

    \subsection{Todo: Exercise 42}
    % https://www.chegg.com/homework-help/bus-driver-pays-tolls-using-nickels-dimes-throwing-one-coin-chapter-8.2-problem-42E-solution-9780073383095-exc
    Show that if $a_n =a_{n-1} + a_{n-2}, a_0=s$ and $a1 =t$, where $s$ and $t$ are constants, then $a_n = sf_{n-1} + tf_n$ for all positive integers $n$.

    \section{Section 5.1} 
    % Obligatoriske: 4, 6, 14, (4, 6, 14); 
    % Anbefalte: 9, 10.

    \subsection{Exercise 4}
    % https://www.chegg.com/homework-help/bus-driver-pays-tolls-using-nickels-dimes-throwing-one-coin-chapter-5.1-problem-4E-solution-9780073383095-exc
    Let $P(n)$ be the statement that $1^3 + 2^3 + \cdots + n^3 = (n(n + 1)/2)^2$ for the positive integer $n$.\\
%     \textbf{a) What is the statement $P(1)$?}\\
%     \textbf{b) Show that $P(1)$ is $true$, completing the basis step of the proof.}\\
%     \textbf{c) What is the inductive hypothesis?}\\
%     \textbf{d) What do you need to prove in the inductive step?}\\
%    \textbf{ e) Complete the inductive step, identifying where you use the inductive hypothesis.}\\
%     \textbf{f) Explain why these steps show that this formula is $true$ whenever $n$ is a positive integer.}
    % 
    % \begin{enumerate}[label=\alph*)]
        
        $P(1): 1^3 = (\frac{1(1+1)}{2})^2$
        Basis step:
            $P(1): 1^3 = (\frac{1(1+1)}{2})^2 = 1$ \\
            $P(1)$ is true, which completes the basis step of a proof by
                induction for $P(k)$
    
        The inductive hypothesis consists of two parts: \\
            - $P(b)$ holds true \\
            - $P(k) \rightarrow P(k+1)$ holds true\\
            Then $P(k), \quad \forall k > b$
    
            In other words, if $P$ is true for the first step, and $P$ holds true for an arbitrary step implies $P$ holds true for the next step, then $P$ holds true for all steps. \\
    
        You need to prove the first step $P(b)$, and then you need to prove
            $P(k) \rightarrow P(k+1)$
    
                    \begin{align}
                        \intertext{Basis step:}
                        1^3 = (1(1+1)/2)^2 = 1 \\
                        \intertext{LHS = RHS}
                        \intertext{Inductive step:}
                        \sum_{n=1}^k{n^3} = \Big(\frac{k(k+1)}{2}\Big)^2 \\ %\label{eq:4e_k}
                        \intertext{We assume $P(k)$ is true for an arbitrary integer $k$. We can replace $k$ with $k+1$. Out goal is to show that if $P(k)$ holds then $P(k+1)$ must hold}
                        \sum_{n=1}^{k+1}{n^3} = \Big(\frac{(k+1)((k+1)+1)}{2}\Big)^2 \\
                        \sum_{n=1}^{k}{n^3} + (k+1)^3 = \Big(\frac{(k+1)(k+2)}{2}\Big)^2 \\
                        = \Big(\frac{(k^2+3k+2)}{2}\Big)^2 \\
                        = \Big(\frac{(k(k+1))}{2} + (k+1)\Big)^2 \\
                        \sum_{n=1}^{k}{n^3} + (k+1)^3 = \Big(\frac{(k(k+1))}{2}\Big)^2 + k(k+1)(k+1) + (k+1)^2 \\
                        \intertext{We subtract  $\sum_{n=1}^k{n^3} = \Big(\frac{k(k+1)}{2}\Big)^2$ from the equation and have}
                        (k+1)^3 = k(k+1)^2 + (k+1)^2 \\
                        (k+1)^3 = (k+1)(k+1)^2 = (k+1)^3 \\
                        \intertext{LHS = RHS. This means that if $P(k)$ holds true,
                        then $P(k+1)$ must also be true, which by the inductive
                        hypothesis means that $\forall k \geq 1, \quad P(k)$ holds true}
                    \end{align}

    \subsection{Exercise 6}
    % https://www.chegg.com/homework-help/bus-driver-pays-tolls-using-nickels-dimes-throwing-one-coin-chapter-5.1-problem-6E-solution-9780073383095-exc
    Prove that $s1 \cdot 1!+2 \cdot 2! + \cdots + n \cdot n! = (n+1)!-1$ whenever $n$ is a positive integer.
    \begin{align}
        \intertext{Basis step:}
        1\cdot 1! = (1 + 1)! - 1 \\
        1 = (2)! - 1 = 2 - 1 = 1\\
        \intertext{LHS = RHS, so $P(1)$ holds true}
        \intertext{Inductive step:}
        \sum_{n = 1}^k{n\cdot n!} = (k + 1)! - 1 \\ %\label{eq:1-6-k}
        \intertext{We assume that $P(k)$ holds for all $k > 1$, and replace
            $k$ with $k+1$}
        \sum_{n = 1}^{k+1}{n\cdot n!} = ((k+1) + 1)! - 1 \\
        (k+1)\cdot(k+1)! + \sum_{n = 1}^{k}{n\cdot n!}  = (k+2)! - 1 \\
        (k+1)\cdot(k+1)! + \sum_{n = 1}^{k}{n\cdot n!}  = (k+2)(k+1)! - 1 \\
        \intertext{We subtract $\sum_{n = 1}^k{n\cdot n!} = (k + 1)! - 1$ from the equation and get}
        (k+1)\cdot(k+1)! +  = (k+2)(k+1)! - 1 - \big( (k+1)! - 1\big)\\
        (k+1)\cdot(k+1)! +  = (k+2)(k+1)!  -  (k+1)! \\
        (k+1)\cdot(k+1)! +  = k(k+1)! + 2(k+1)! - (k+1)! \\
        (k+1)\cdot(k+1)! +  = k(k+1)! + (k+1)!  \\
        (k+1)\cdot(k+1)! +  = (k+1)\cdot(k+1)!  \\
        \intertext{LHS = RHS, so $P(k)$ must hold for all $k > 1$}
    \end{align}


    \subsection{Exercise 14}
    % https://www.chegg.com/homework-help/bus-driver-pays-tolls-using-nickels-dimes-throwing-one-coin-chapter-5.1-problem-14E-solution-9780073383095-exc
    Prove that for every positive integer $n$, $\sum_{k=1}^{n}k2^k=(n-1)2^{n+1}+2$.
            \begin{align}
                \intertext{$P(k)$:}
                \sum_{n=1}^k{ n2^n} = (k-1)2^{k+1} + 2 %\label{eq:1-14-k}\
                \intertext{Basis step:}
                1\cdot 2^1 = (1-1)2^2{1+1} + 2 = 2\
                \intertext{LHS = RHS, so $P(1)$ holds true}
                \intertext{Induction step:}
                \intertext{Replacing $n$ with $k+1$ in $\sum_{n=1}^k{ n2^n} = (k-1)2^{k+1} + 2$ gives us}
                \sum_{n=1}^{k+1}{ n2^n} = ((k+1)-1)2^{(k+1)+1} + 2 \\
                (k+1)2^{k+1} + \sum_{n=1}^{k}{ n2^n} = k2^{k+2} + 2 \\
                (k+1)2^{k+1} + \sum_{n=1}^{k}{ n2^n} = 2k2^{k+1} + 2 \\
                \intertext{We subtract $\sum_{n=1}^k{ n2^n} = (k-1)2^{k+1} + 2$ from the equation and get}
                (k+1)2^{k+1} = 2k2^{k+1} + 2 - \big((k-1)2^{k+1} + 2\big) \\
                (k+1)2^{k+1} = 2k2^{k+1} - (k-1)2^{k+1} \\
                (k+1)2^{k+1} = (2k - (k-1)) 2^{k+1}\\
                (k+1)2^{k+1} = (k+1)2^{k+1}\\
                \intertext{LHS = RHS, so $P(k)$ must hold for all $k > 1$}
            \end{align}
\end{document}