\documentclass[12pt]{article}
    \usepackage{mathtools}
    \usepackage[hidelinks]{hyperref}
    \usepackage{color}
    \usepackage{fancyref}
    \usepackage{lastpage}
    \usepackage{fancyhdr}

    \pagestyle{fancy}
    \fancyhf{}
    \fancyhead[LO]{TMA4140: Homework Set 3}
    \fancyhead[RO]{Henry S. Sjøen}
    \fancyfoot[CO]{\thepage\ of \pageref{LastPage}}

    \definecolor{darkred}{RGB}{200, 0, 0}

  \author{Henry S. Sjøen}
  \title{
  \textbf{TMA4140 - Homework Set 5}\\
  Basic structures: Sets, Functions, Sequences and Sums\\
    \color{darkred}{\textbf{RETTES}}
  }
\begin{document}
    \maketitle
    \thispagestyle{empty}
    \pagebreak
    \tableofcontents
    \pagebreak
    
    \section{Chapter 4.4}
    \subsection{Exercise 21}
    Use the construction in the proof of the Chinese remainder theorem to find all solutions to the system of congruences $x \equiv 1(mod 2)$, $x \equiv 2 (mod 3)$, $x \equiv (mod 5)$, and $x \equiv 4 (mod 11)$.
  
    First congruence can be written as:
    \begin{equation}
      \begin{split}
        x \equiv  1(mod2)\\
        x = 2m + 1
      \end{split}
    \end{equation}
    
    Second congruence can be written as:
    \begin{equation}
      \begin{split}
        x \equiv  2(mod3)\\
        x = 3m + 2
      \end{split}
    \end{equation}
    
    Third congruence can be written as:
    \begin{equation}
      \begin{split}
      x \equiv  2(mod5)\\
      x = 3m + 2
      \end{split}
    \end{equation}

    And the final congruence can be written as:
    \begin{equation}
      \begin{split}
        x \equiv 4(mod11)\\
        x = 11m+4
      \end{split}
    \end{equation}

    Using the Chinese theorem, we can find the L.C.D for $2,3,5$ and $11$.
  
    $LCD = 2x3x5x11=330$

    Thus, $x = 330k+I$ where $I$ is $1,2,3,4$ for $mod 2,3,5,11$.

    Then $8,13,18,K,323,K$ are numbers divided by 5 where we get remainder 3.

    TODO: finish this (step 4)
    (Step 5)

    Thus the remainder is 1.
    Thus the remainder is 2.
    Thus the remainder is 4.

    Hence, the solution of the provided system of congruences is $x = 232 + 330k$.
    
    \subsection{Exercise 33}
    Use Fermat’s little theorem to find $7^{121} mod 13$.

    \begin{figure}[h]
      \label{theorem:fermat}
      \centering    
      Fermat’s theorem\\
      If $p$ is prime and $a$ is an integer not divisible by $p$, then $ap-1\equiv 1 (mod p)$.\\
      % Furthermore, for every integer a we have $ap \equiv a (mod p)$.
    \end{figure}

By Fermat’s theorem,
$7^{12} \equiv 1(mod13)$
So to find $7^{121}(mod13)$ we have to compute
\begin{equation}
  \begin{split}
    7^{121}&=(7^{12})^{10}\cdot 7(mod13)\\
    &\equiv (7^{12})^{10}(mod13)\cdot 7(mod13)\\
    &\equiv 1(mod13)\cdot7(mod13)\\
    &\equiv 7(mod13)
  \end{split}
\end{equation}
Answer is $7(mod13)$

\subsection{Exercise 37a}
    Show that $2^{340} \equiv 1 (mod 11)$ by Fermat’s little theorem and noting that $2^{340} = (2^{10})^{34}$.

    If $p$ is prime and a is an integer not divisible by p, then
    $a^{p-1}\equiv 1(mod p)$.


    So $2^{10}\equiv 1(mod11)$
    \begin{equation}
      \begin{split}
        2^{340}&=(2^{10})^{34}\\
        &\equiv(2^{10})^{34}(mod11)\\
        &\equiv 1(mod11)
      \end{split}
    \end{equation}

    \section{Chapter 4.5}
    \subsection{Exercise 12}
    Find the sequence of pseudorandom numbers generated by the power generator with $p = 11$, $d=2$, and seed $x_0=3$.


    % The power generator is the method of generating pseudorandom numbers

    Let's use the formulae
    \begin{figure}
      \centering
      \begin{equation}
      \begin{split}
        x_{n+1}=x_n^d mod p\\
        x_{n+1}=x_n^2 mod 1 1
      \end{split}
    \end{equation}
    Where $p=11,d=2, x_0=3$
\end{figure}
  
    So the sequence can be generated as
    \begin{equation}
      \begin{split}
        x_1=3^2mod1 1\\
        \equiv 9 mod 1 1\\
        x_2=9^2mod1 1\\
        \equiv 4 mod 1 1\\
        x_2=4^2mod1 1\\
        \equiv 5 mod 1 1\\
        x_2=5^2mod1 1\\
        \equiv 3 mod 1 1\\
      \end{split}
    \end{equation}
    Then the sequence repeats...
    So the Sequence is $3,9,4,5,3,9,4,5...$

    \section{Chapter 6.1}
    \subsection{Exercise 27}
    A committee is formed consisting of one representative from each of the 50 states in the United States, where the representative from a state is either the governor or one of the two senators from that state. How many ways are there to form this committee?

    The number of different possible committees are $3^{50}=7,178979877e23$

    \subsection{Exercise 44}
    How many ways are there to seat four of a group of ten people around a circular table where two seatings are considered the same when everyone has the same immediate left and immediate right neighbor?

    The number of ways of seating a group of 4 people from the 10 people are
    \begin{equation}
      \begin{split}
        10*9*8*7 =5040
      \end{split}
    \end{equation}

    But our number should be less than this because we don't count all the possible arrangements. We here get 4 of the same seating arrangements that we have remove...
    The required number of ways are therefore
    \begin{equation}
      \begin{split}
        \frac{5040}{4}=1260
      \end{split}
    \end{equation}

    \section{Chapter 6.2}
    \subsection{Exercise 10}
    Let $(x_i,y_i), i = 1,2,3,4,5$,be a set of five distinct points with integer coordinates in the $xy$ plane. Show that the midpoint of the line joining at least one pair of these points has integer coordinates.

    The middle point of two co-ordinates $(x_i,y_i)$ and $(x_j,y_j)$ is $(\frac{x_i+x_j}{2},\frac{y_i+y_j}{2})$. For the co-ordinates mid-point to be an integer,  $(x_i,y_i)$ and $(x_j,y_j)$ needs to be even number and hence should have the same parity.

    Similarly, $y_i$ and $y_j$ have same parit. There are four possible cases of the parity of pair of integers: $(odd,odd),(odd,even),(even,odd),(even,even)$.

    By \textit{pigeon hole principle}, since there are four integer pairs, at least two of them needs to have same parity and hence will have mid-point having integer coordinates.

    \subsection{Exercise 16}
    How many numbers must be selected from the set ${1, 3, 5, 7, 9, 11, 13, 15}$ to guarantee that at least one pair of these numbers add up to 16?
    Best case only two: $1+15=16$ but worst case?

    % TODO: SHOW SOLUTION?
    To completely guarantee that at least one pair of the numbers add up to 16, we need to pick at least 5 numbers from the set.

    \subsection{Exercise 18}
    Suppose that there are nine students in a discrete mathematics class at a small college. We can list all possible (Male, Female)-combinations as shown in \fref{fig:groupings}.
   \begin{figure}[h]
    \label{fig:groupings}
    \begin{equation}
      \begin{split}
        (0,9),(1,8),(2,7),(3,6),(4,5),(5,4)(6,3),(7,2),(8,1),(9,0)
      \end{split}
    \end{equation}
    \centering
    In the order (Male, Female)
  \end{figure}
   
    \textbf{a) Show that the class must have at least five male students or at least five female students.}\\
    Considering only two kind of genders, Male and Female. If there are 3 male, then the remaning 6 students must be female and vise versa.

    \textbf{b) Show that the class must have at least three male students or at least seven female students.}\\
    Same as for the exercise above. If 2 male, then the remaning 7 must be female. If more than 2 male, then the statement is still valid.
    
    \section{Chapter 6.3}
    A coin is flipped 10 times where each flip comes up either heads or tails. How many possible outcomes...

    \subsection{Exercise 19a}
    ...are there in total?\\
    $2^{10}=1024$

    \subsection{Exercise 19b}
    ...contain exactly two heads?\\
    $C(10,2)=45$

    \subsection{Exercise 19c}
    ...contain at most three tails?
    \begin{equation}
      \begin{split}
        &C(10,1)+C(10,2)+C(10,3)\\
        &=1+10+45+120\\
        &=176        
      \end{split}
    \end{equation}

\end{document}