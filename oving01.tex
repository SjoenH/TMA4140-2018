\documentclass[12pt]{article}
  \usepackage{mathtools}

\author{Henry S. Sjoen}
\title{TMA4140: Homework Set 1}

    \begin{document}
    \maketitle
    \tableofcontents
\title{}
    \section{1.1}
    \subsection{12c}
You miss the final examination therefore you do not pass the course.
\subsection{12f}
You have the flu and you miss the final examination or you don't have the flu and you pass the course.
\subsection{14}
p: You get an A on the final. \newline
q: You do every exercise in this book. \newline
r: you get an A in this class.

\subsubsection{14.a}

\begin{math}
  r \wedge \neg q
\end{math}

\subsubsection{14.e}
\begin{math}
  (p \wedge q) \rightarrow r
\end{math}
\section{1.3}
\subsection{10}
\subsubsection{10.a}
\begin{table}[tbh]
  \begin{tabular}{cccccc}
    p  & q  & $\neg p$ & $p \wedge q$  & $(\neg p \wedge (p \vee q))$ &$(\neg p \wedge (p \vee q)) \rightarrow q$  \\
    0  & 0  & 1        & 0 & 0 &1\\
    0  & 1  & 1        & 0 & 0 &1\\
    1  & 0  & 0        & 0 & 0 &1\\
    1  & 1  & 0        & 1 & 0 &1\\ 
  \end{tabular}
\end{table}
Proven: It's a Tautology!
\subsubsection{10.b}
\begin{table}[tbh]
  \begin{tabular}{cccccccc}
    p  & q  & r &$p \rightarrow q$ & $q \rightarrow r$&$p \rightarrow q \wedge q \rightarrow r$&$p \rightarrow r$&$(p \rightarrow q \wedge q \rightarrow r) \rightarrow (p \rightarrow r) $\\
    0& 0& 0& 1& 1& 1& 1& 1\\
    1& 0& 0& 0& 1& 0& 0& 1\\
    0& 1& 0& 1& 0& 0& 1& 1\\
    1& 1& 0& 1& 0& 0& 0& 1\\
    0& 0& 1& 1& 1& 1& 1& 1\\
    1& 0& 1& 0& 1& 0& 1& 1\\
    0& 1& 1& 1& 1& 1& 1& 1\\
    1& 1& 1& 1& 1& 1& 1& 1
  \end{tabular}
\end{table}

Proven: It's a Tautology!

\subsubsection{10.c}

\begin{math}
  [p \wedge (p\rightarrow q)] \rightarrow q
\end{math}

\begin{table}[tbh]
  \begin{tabular}{ccccc}
    p & q& $p\rightarrow q$&$p\wedge(p\rightarrow q)$&$[p\wedge(p\rightarrow q)]\rightarrow q$\\
    0& 0& 1& 0& 1\\
    0& 1& 1& 0& 1\\
    1& 0& 0& 0& 1\\
    1& 1& 1& 1& 1
  \end{tabular}
\end{table}

Proven: It's a Tautology!

\subsubsection{10.d}
\begin{table}[p]
  \begin{tabular}{cccccccc}
    p& q& r& $p\vee q$& $p\rightarrow r$& $q\rightarrow r$& $(p\vee q) \wedge (p\rightarrow r) \wedge (q\rightarrow r)$&$[(p\vee q) \wedge (p\rightarrow r) \wedge (q\rightarrow r)] \rightarrow r $ \\
    0& 0& 0& 0& 1& 1 &0 &1 \\
    0& 0& 1& 0& 1& 1 &0 &1 \\
    0& 1& 0& 1& 1& 0 &0 &1 \\
    0& 1& 1& 1& 1& 1 &1 &1 \\
    0& 0& 0& 0& 1& 1 &0 &1 \\
    0& 0& 1& 0& 1& 1 &0 &1 \\
    1& 1& 0& 1& 0& 0 &0 &1 \\
    1& 1& 1& 1& 1& 1 &1 &1 \\
    1& 0& 0& 1& 0& 1 &0 &1 \\
    1& 0& 1& 1& 1& 1 &1 &1 \\
    1& 1& 0& 1& 0& 0 &0 &1 \\
    1& 1& 1& 1& 1& 1 &1 &1
  \end{tabular}
  \caption{Table for Exercise 10d}
  \label{table:10d}
\end{table}

Proven: It's a Tautology! See table \ref{table:10d} on page \pageref{table:10d}.

\section{1.4}
\subsection{24d}
All students in your class can solve quadratic equations.

1. Domain: People in my class.

\begin{math}
  \forall{x} Q(x)
\end{math}

2. Domain: All People

$P(x)$: $x$ in my class
$Q(x)$: $x$ can solve quadratic equations

\begin{math}
  \forall{x}(P(x)\rightarrow Q(x))
\end{math}

\subsection{24e}
$R(x)$: $x$ want to be rich.

1. Domain: People in my class.

$\exists{x} \neg R(x)$

2. Domain: All People

$\exists{x} (P(x) \wedge \neg R(x))$

\section{1.5}
\subsection{12b}
Rachel has not chatted with Chelsea.

Can be expressed as
\begin{math}
  \neg C(Rachel,Chelsea)
\end{math}

Where C(x,y) is the relation, \emph{x have chatted with y}.

\subsection{12e}

\begin{math}
  \forall{y} C(Sanjay, y) \wedge \neg C(Sanjay,Joseph)
\end{math}

\subsection{30c}

\begin{math}
  \neg \exists{y}(Q(y)) \wedge \neg \exists{y} (\forall{x}\neg R(x,y)) \newline
  \forall{y}[\neg Q(y) \wedge \neg \forall{x} \neg R(x,y)]\newline
  \forall{y}[\neg Q(y) \wedge \exists{x} R(x,y)]
\end{math}

\subsection{30e}
\begin{math}
  \neg \exists{y}[\forall{x}\exists{z}T(x,y,z)\vee \exists{x}\forall{z}U(x,y,z)] \newline
  \forall{y} [\neg\forall{x}\exists{z}T(x,y,z)\wedge \neg \exists{x}\forall{z}U(x,y,z)] \newline
  \forall{y}(\exists{x}\forall{z} \neg T(x,y,z) \wedge \forall{x}\exists{z}\neg U(x,y,z)
\end{math}
\end{document}