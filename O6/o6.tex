\documentclass[12pt]{article}
    \usepackage{mathtools}
    \usepackage[hidelinks]{hyperref}
    \usepackage{color}
    \usepackage{fancyref}
    \usepackage{lastpage}
    \usepackage{fancyhdr}
    
    \usepackage{pagecolor}
    % \pagecolor{black}
    % \color{white}

    \pagestyle{fancy}
    \fancyhf{}
    \fancyhead[LO]{TMA4140: Homework set 6}
    \fancyhead[RO]{Henry S. Sjøen \& Toralf Tokheim}
    \fancyfoot[CO]{\thepage\ of \pageref{LastPage}}

    \definecolor{darkred}{RGB}{200, 0, 0}

  \author{Henry S. Sjøen \& Toralf Tokheim}
  \title{
  \textbf{TMA4140 - Homework Exercise Set 6}}
\begin{document}
    \maketitle
    \thispagestyle{empty}
    \pagebreak
    \tableofcontents
    \pagebreak


    \section{Chapter 6.3} 
    \subsection{Exercise 13}
    A group contains $n$ men and $n$ women. How many ways are there to arrange these people in a row if the men and women alternate?
    % TODO fill in some info?
    Total number of people in the group is $n$-men and $n$-women therefore $2n$ in total.
    For men we have: $n! \times n!=(n!)^2$ and it's the same for women.
    Therefore when combined we get the total number of arrangements, which are $(n!)^2+(n!)^2=2(n!)^2$ if the row alternates between men and women.

    \subsection{Exercise 34}
    Suppose that a department contains 10 men and 15 women. How many ways are there to form a committee with six members if it must have more women than men?

    3 possible type of ways to form a committee:\\
    Possibility number 1: 6 women and 0 men gives us $5005$ ways. 
    \begin{equation}
        C(15,6)\times C(10,0)=5005
    \end{equation}
    Possibility number 2: 5 women and 1 man gives us $30030$ ways.
    \begin{equation}
        C(15,5)\times C(10,1)=30030
    \end{equation}
    Possibility number 3: 4 women and 2 men gives us $61425$ ways.
    \begin{equation}
        C(15,4)\times C(10,2)=61425
    \end{equation}

    So to combine all the possible ways to form a committee we get:
\begin{equation}
    \begin{split}
        C(15,6)\times C(10,0)\times C(15,5)\times C(10,1)\times C(15,4)\times C(10,2)\\
        =5005+30030+61425=96460
    \end{split}
\end{equation}

    \section{Chapter 6.4} 
    \subsection{Exercise 4}
    Find the coefficient of $x^5y^8$ in $(x+y)^{13}$.

    \begin{figure}
        \begin{equation}
            \begin{split}
                (x+y)^n =&\sum_{j=0}^{n}\binom{n}{j} x^{n-j}y^j\\=& \binom{n}{0}x^n+\binom{n}{1}x^{n-1}y+...+\binom{n}{n-1}xy^{n-1}+\binom{n}{n}y^n
            \end{split}
        \end{equation}
    \end{figure}
    
    
    From the binomial theorem it follows that this coefficient is
    \begin{equation}
        \begin{split}
            \binom{13}{8}= 
            \frac{13!}{8!(13-8)!}=
            \frac{13!}{8!5!}=
            1 287
        \end{split}
    \end{equation}

    \subsection{Exercise 9} 
    What is the coefficient of $x^{101}y^{99}$ in the expansion of $(2x-3y)^{200}$?
    
    \begin{equation}
        \begin{split}
            \binom{200}{99}*2^{101}*(-3)^{99}=& -(2^{101}*3^{99}*\binom{200}{99}) \\
            =& - (2^{101}*2^{99}\frac{200!}{99!(200-99)!}) \\
            =& - (2^{101}*2^{99}\frac{200!}{99!101!}) 
        \end{split}
    \end{equation}


    \subsection{Exercise 12}
    The row of Pascal's triangle containing the binomial coefficients ${10 \choose k}, 0 \leq k \leq 10,$ is: $1, 10, 45, 120, 210, 252, 210, 120, 45, 10, 1$.

    Use Pascal's identity to produce the row immediately following this row in Pascal's triangle.
    \begin{equation}
        \begin{split}
            \binom{n+1}{k}=\binom{n}{k-1}+\binom{n}{k}
        \end{split}
    \end{equation}
    Setting $=10$ and $k=1$.
    \begin{equation}
        \begin{split}
            \binom{10+1}{1}&=\binom{10}{1-1}+\binom{10}{1}\\
            \frac{11!}{1!(10)!}&=\frac{10!}{0!(10)!}+\frac{10!}{1!(9)!}\\
            11 &= 1+10\\
            11 &= 11
        \end{split}
    \end{equation}
    Pascal's identity is verified.\\
    Using the binomial coefficients $\binom{n+1}{k}$  to find the immediate following row in Pascal's triangle.
    
   Substituting $n=10$ and $k = 0$ in $\binom{n+1}{k}$ 
    \begin{equation}
        \begin{split}
            \binom{10+1}{0}=\binom{11}{0}=1
        \end{split}
    \end{equation}
   

    Substituting $n=10$ and $k = 1$ in $\binom{n+1}{k}$ to find the immediately following number in the row... and so on.
    \begin{equation}
        \begin{split}
            \binom{10+1}{1}=\binom{11}{1}=11
        \end{split}
    \end{equation}


    Substituting $n=10$ and $k = 2$ 
    \begin{equation}
        \begin{split}
            \binom{10+1}{2}=\binom{11}{2}=55
        \end{split}
    \end{equation}


    Substituting $n=10$ and $k = 3$ 
    \begin{equation}
        \begin{split}
            \binom{10+1}{3}=\binom{11}{3}=165
        \end{split}
    \end{equation}


    Substituting $n=10$ and $k = 4$ 
    \begin{equation}
        \begin{split}
            \binom{10+1}{4}=\binom{11}{4}=330
        \end{split}
    \end{equation}


    Substituting $n=10$ and $k = 5$ 
    \begin{equation}
        \begin{split}
            \binom{10+1}{5}=\binom{11}{5}=462
        \end{split}
    \end{equation}


    Substituting $n=10$ and $k = 6$ 
    \begin{equation}
        \begin{split}
            \binom{10+1}{6}=\binom{11}{6}=462
        \end{split}
    \end{equation}


    Substituting $n=10$ and $k = 7$ 
    \begin{equation}
        \begin{split}
            \binom{10+1}{7}=\binom{11}{7}=330
        \end{split}
    \end{equation}


    Substituting $n=10$ and $k = 8$ 
    \begin{equation}
        \begin{split}
            \binom{10+1}{8}=\binom{11}{8}=165
        \end{split}
    \end{equation}


    Substituting $n=10$ and $k = 9$ 
    \begin{equation}
        \begin{split}
            \binom{10+1}{9}=\binom{11}{9}=55
        \end{split}
    \end{equation}


    Substituting $n=10$ and $k = 10$ 
    \begin{equation}
        \begin{split}
            \binom{10+1}{10}=\binom{11}{10}=11
        \end{split}
    \end{equation}


    Substituting $n=10$ and $k = 11$ 
    \begin{equation}
        \begin{split}
            \binom{10+1}{11}=\binom{11}{11}=1
        \end{split}
    \end{equation}
    
    Which gives us the row 
    \begin{equation}
        \begin{split}
            [\binom{11}{0},\binom{11}{1},\binom{11}{2},\binom{11}{3},\binom{11}{4},\binom{11}{5},\binom{11}{6},\binom{11}{7},\binom{11}{8},\binom{11}{9},\binom{11}{10},\binom{11}{11}]\\
            =[1, 11, 55, 165, 330, 462, 462, 330, 165, 55, 11, 1]   
        \end{split}
    \end{equation}
    
    \section{Chapter 6.5} 
    \subsection{Exercise 6}
    How many ways are there to select five unordered elements from a set with three elements when repetition is allowed?

    There are $C(n+r-1,r)=C(n+r-1,n-1)r$-combinations from a set with $n$ elements when repetition is allowed.
    Setting $n=3$ and $r=5$.
    \begin{equation}
            C(3+5-1,5) = C(7,5) 
    \end{equation}

    Since $C(n,r)=C(n,n-r)$, we get...
    \begin{equation}
    C(7,2) = \frac{7\times 6}{2} =\frac{42}{2} = 21
  \end{equation} 

    There are therefore 21 different ways to select 5 unordered elements from a set of 3 elements when repetition is allowed.
    % ${n \choose k} = {3 \choose 5}$?

    \subsection{Exercise 14} 
    How many solutions are there to the equation $x_1 + x_2 + x_3 + x_4 = 17$, where $x_1,x_2,x_3$, and $x_4$ are nonnegative integers?

    There are $C(n+r-1,r)=C(n+r-1,n-1)r$-combinations from a set with $n$ elements when repetition is allowed.

    Select 17 items from a set with four elements...
    Set $n = 4$ and $r=17$.
    \begin{equation}
        \begin{split}
            C(4+17-1,17)&=C(4+17-1,3)\\
            &=C(20,3)\\
            &=1140
        \end{split}
    \end{equation}

    There are $1140$ Solutions.

    \subsection{Exercise 30} 
    How many different strings can be made from the letters in \textit{MISSISSIPPI}, using all the letters?

    There are 11 number of letters, so... $n=11$. 
    There are 4 unique letters, therefore...$k=4$.

    \begin{equation}
        \begin{split}
            \frac{n!}{n_1!n_2!...n_k!}=\frac{11!}{1!4!4!2!}=34650
        \end{split}
    \end{equation}

    We can form $34650$ different strings from the letters.

    \subsection{Exercise 54}
    How many ways are there to distribute five indistinguishable objects into three indistinguishable boxes?

    Assuming we can have empty boxes, Then we can have the following distinct arrangements:
    \begin{equation}
        \begin{split}
            [0][0][5]\\
            [0][1][4]\\
            [0][2][3]\\
            [1][1][3]\\
            [1][2][2]
        \end{split}
    \end{equation}

    So there are 5 distinct orderings / total number of partitions.

    \section{Chapter 6.6} 
    \subsection{Exercise 5a}
    Find the next larger permutation in lexicographic order after the permutation of $1432$.
    Next order is $2134$.


    \subsection{Exercise 5c}
    Find the next larger permutation in lexicographic order after the permutation of $12453$.
    Next order is $12534$.

\end{document}