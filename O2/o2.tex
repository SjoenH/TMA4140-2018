\documentclass[12pt]{article}
    \usepackage{mathtools}
    \usepackage{amsfonts}

    \usepackage{venndiagram}
    \usepackage{lastpage}
    \usepackage[hidelinks]{hyperref}
    \usepackage{color}
    \definecolor{darkred}{RGB}{200, 0, 0}

    \usepackage{fancyhdr}
    \pagestyle{fancy}
    \fancyhf{}
    \fancyhead[LO]{TMA4140: Homework Set 2}
    \fancyhead[RO]{Henry S. Sjøen}
    \fancyfoot[CO]{\thepage\ of \pageref{LastPage}}
         
    \usepackage{fancyref}

  \author{Henry S. Sjøen}
  \title{%
  \textbf{TMA4140 - Homework Set 2}\\
  Basic structures: Sets, Functions, Sequences and Sums\\
    \color{darkred}{\textbf{RETTES}}
  }
    \begin{document}
    \maketitle 
    \thispagestyle{empty}
    % \pagebreak
    \tableofcontents
      
    \pagebreak
    \section{Chapter 2.1 - Sets}
    \subsection{Exercise 5}
    Determine whether each of these pairs are equal.\\
        a) $\{1,3,3,3,5,5,5,5\},\{5,3,1\} \Rightarrow True$.\\
        b) $\{\{1\}\},\{1,\{1\}\} \Rightarrow False.$\\
        c) $\emptyset,\{\emptyset\} \Rightarrow False$
   
    \subsection{Exercise 24}
    Determine wether each of these sets is the power set of a set, where \emph{a} and \emph{b} are distinct elements.\\
        a) $\emptyset$, is False. $P(\emptyset)=\{\emptyset\{\emptyset\}\}$\\ 
        b) $\{\emptyset,\{a\}\}$, is True. $P(\{a\})=\{\emptyset,\{a\}\}$\\
        c) $\{\emptyset, \{a\},\{\emptyset,a\}\}$, is False.\\
        d) $\{\emptyset,\{a\},\{b\},\{a,b\}\}$, is True.

        The powerset of any set $S$ is the set of all subsets of $S$, including the empty set $\emptyset$ and $S$ itself. \footnote{Wikipedia, 2018-09-14, 13:54, \url{https://en.wikipedia.org/wiki/Power_set}}
        If $a$ and $b$ are distinct elements of a set $A$. $A=\{a,b\}$
        Then the powerset of $A$ is $p(A)=\{\emptyset,\{a\},\{b\},\{a,b\}\}$.

    \pagebreak
    \section{Chapter 2.2 - Set Operations}
    \subsection{Exercise 18c}
    Let A, B and C be sets. Show that:
    $(A \cap B) \subseteq (A \cup B \cup C)$
    
    % \begin{equation}
    %     \begin{split}
    %         (A \cap B) \subseteq (A \cup B \cup C)
    %     \end{split}
    % \end{equation}
    
    \begin{figure}[h]
        \centering
        \begin{venndiagram3sets}
            \fillA
            \fillB
            \fillC
        \end{venndiagram3sets}    
        \caption{$(A \cup B \cup C)$}
        \label{fig:Union}
    \end{figure}
    
    As illustrated in \fref{fig:Union} we can se the venndiagram for $(A \cup B \cup C)$. Coloring out $(A \cap B)$, as shown in  \fref{fig:ACapB}, we can see that $(A \cap B)$ is a subset of $(A \cup B \cup C)$.
    
    \begin{figure}[h]
        \centering
        \caption{$A \cap B$}
        \begin{venndiagram3sets}
            % [labelOnlyAB=$A \cap B$]
            \fillACapB
        \end{venndiagram3sets}
        \label{fig:ACapB}
    \end{figure}

        % \url{http://www3.wolframalpha.com/Calculate/MSP/MSP2891521ibi0a5gd30f95g000037hcb6a7hd034233?MSPStoreType=image/gif&s=44&w=471.&h=279.&cdf=Resizeable}

    \newpage
    \subsection{Exercise 18d}
    Let A, B and C be sets. Show that:
    $(A - B) - C \subseteq A - C$


    \begin{figure}[h]
        \centering
        \begin{venndiagram3sets}
            \fillOnlyA
        \end{venndiagram3sets}
        \caption{$(A - B) - C$}
        \label{fig:onlyA}
    \end{figure}
    
    \begin{figure}[h]
        \centering
        \begin{venndiagram3sets}
            \fillANotC
        \end{venndiagram3sets}
        \caption{$A - C$}
        \label{fig:ANotC}
    \end{figure}    

    We can see that $(A - B)-C$ is a subset of $A-C$, see \fref{fig:onlyA} and \ref{fig:ANotC}.
    
    \newpage
    \subsection{Exercise 46}
    Show that if A, B, and C are sets, then:\footnote{This is a special case of the inclusion-exclusion principle, which will be studied in Chapter 8. Also... Wikipedia \url{https://en.wikipedia.org/wiki/Inclusion-exclusion_principle}}
    
    \begin{equation}
        |A \cup B \cup C| = 
        |A| + |B| + |C| - |A \cap B| - |A \cap C| - |B \cap C| + |A \cap B \cap C|
    \end{equation}

    \begin{figure}[h]
        \centering
        \begin{venndiagram3sets}[
            labelOnlyA=1,labelOnlyB=1,labelOnlyC=1,
            labelOnlyAB=2,labelOnlyAC=2,labelOnlyBC=2,
            labelABC=3]
            \fillA
            \fillB
            \fillC
        \end{venndiagram3sets}
        \caption{$|A|+|B|+|C|$}
        \label{fig:u1}
    \end{figure}    

    Here we have counted some elements more than once (\fref{fig:u1}), lets correct that.

    \begin{figure}[h]
        \centering
        \begin{venndiagram3sets}[
            labelOnlyA=1,labelOnlyB=1,labelOnlyC=1,
            labelOnlyAB=1,labelOnlyAC=1,labelOnlyBC=1,
            labelABC=0]
            \fillA
            \fillB
            \fillC
        \end{venndiagram3sets}
        \caption{$|A|+|B|+|C|-(|A \cap B| - |A \cap C| - |B \cap C|)$}
        \label{fig:u2}
    \end{figure}    

    But now, the intersection of A,B and C is not counted. (\fref{fig:u2}). Lets count the intersection once $|A\cap B \cap C|$ as shown in \fref{fig:u3}.

    \begin{figure}[h]
        \centering
        \begin{venndiagram3sets}[
            labelOnlyA=1,labelOnlyB=1,labelOnlyC=1,
            labelOnlyAB=1,labelOnlyAC=1,labelOnlyBC=1,
            labelABC=1]
            \fillA
            \fillB
            \fillC
        \end{venndiagram3sets}
        \caption{$|A|+|B|+|C|-(|A \cap B| - |A \cap C| - |B \cap C|) + |A \cap B \cap C|$}
        \label{fig:u3}
    \end{figure}    

    We have now shown that 
    and we can intuitively see that it's the same as counting all elements inside the union of the three sets: $|A \cup B \cup C|$. (\fref{fig:u4})

    \begin{figure}[h]
        \centering
        \begin{venndiagram3sets}
            \fillA
            \fillB
            \fillC
        \end{venndiagram3sets}
        \caption{Union of the sets A,B and C}
        \label{fig:u4}
    \end{figure}

    % TODO: Show it with more amazing math.

    \pagebreak     
    \section{Chapter 2.3 - Functions}
    \subsection{Exercise 12c}
    Determine whether each of these functions from $Z$ to $Z$ is one-to-one.

    \textbf{True.} $f(n) = n^3$, is One-to-One, because it passes both the vertical and horizontal line test.\footnote{\url{http://www.mathwords.com/o/one_to_one_function.htm}}

    \subsection{Exercise 38}
    % Todo: Look over later
    Let $f(x) = ax+b$ and $g(x)= cx+d$, where a,b,c, and d are constants. Determine necessary and sufficient conditions on the constants a,b,c, and d so that 
    $f \cdot g = g \cdot f$.
           
    \begin{equation}
        \begin{split}
            f \cdot g = g \cdot f = a(cx+d)+b=acx+ad+b\\
            g\cdot f = g(f(x))=c(ax+b)+d=acx+cb+d\\
            f \cdot g =  g\cdot f \Leftrightarrow ad+b = cb+d
        \end{split}
    \end{equation}

    \subsection{Exercise 42}
    % Todo: Look over later
    Let $f$ be the function from $R$ to $R$ defined by $f(x)=x^2$. Find:\\
    a) $f^{-1}(\{1\}) = \pm 1$\\
    b) $f^{-1}(\{ x | 0 < x < 1\}) = \pm \{x|-1<x<1 \wedge x \neq 0 \}$\\
    c) $f^{-1}(\{ x | x > 4\}) = \{x|-2>x \wedge x>2\}$
    
    \pagebreak     
    \section{Chapter 2.4 - Sequences and Summations}

    \subsection{Exercise 12c}
    % Todo: Look over later    
    Show that the sequence {$a_n$} is a solution of the recurrence relation $a_n=-3a_{n-1}+4a_{n-2}$ if $a_n = (-4)^n$
    \begin{equation}
        \begin{split}
            a_n =& -3a_{n-1}+4a_{n-2} = (-4)^n\\
            =&-3a_{n-1} + 4a_{n-2}\\
            =& (-4)^{n-1}+4(-4)^{n-2}\\
            =& (-4)^{n-2}[(-3)(-4)+4]\\
            =& (-4)^{n-2}(16)\\
            =& (-4)^{n-2}(-4)^2\\
            =& (-4)^{n}
        \end{split}
    \end{equation}

    \subsection{Exercise 33d}
    Compute the double sum
    \begin{equation}
        \begin{split}
            \sum_{i=0}^{2}\sum_{j=1}^{3}ij &= (0*1+0*2+0*3)+(1*1+1*2+1*3)+(2*1+2*2+2*3) \\
            &= (0+0+0) + (1+2+3) + (2+4+6)\\ 
            &= 0 + 6 + 12\\ 
            &= 18 
        \end{split}
    \end{equation}

    \pagebreak     
    \section{Chapter 2.5 - Cardinality of Sets}
    \subsection{TODO: Exercise 16}
    \emph{Exercise}: 
    Show that a subset of a countable set is also countable.\\
    \emph{Answer}:
    A set is countable if it is finite or is the same size as $\mathbb{N}$.
    To show that $A$ is countable, it is sufficient to show that there is an injection
    from $A$ to $\mathbb{N}$.

    \end{document}