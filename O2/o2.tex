\documentclass[12pt]{article}
    \usepackage{mathtools}
    \usepackage[hidelinks]{hyperref}
    \usepackage{color}
    \definecolor{darkred}{RGB}{200, 0, 0}

    \usepackage{fancyhdr}
    \pagestyle{fancy}
    \fancyhf{}
    \fancyhead[LO]{TMA4140: Homework Set 2}
    \fancyhead[RO]{Henry S. Sjøen}
    \fancyfoot[RO]{\thepage}
         
  \author{Henry S. Sjøen}
  \title{%
  \textbf{TMA4140 - Homework Set 2}\\
  Basic structures: Sets, Functions, Sequences and Sums\\
    \color{darkred}{\textbf{RETTES}}
  }
    \begin{document}
    \maketitle \thispagestyle{empty}
    % \pagebreak
    \tableofcontents
      
    \pagebreak
    \section{Chapter 2.1 - Sets}
    \subsection{Exercise 5}
    Determine whether each of these pairs are equal.\\
        a) $\{1,3,3,3,5,5,5,5\},\{5,3,1\} \Rightarrow True$.\\
        b) $\{\{1\}\},\{1,\{1\}\} \Rightarrow False.$\\
        c) $\emptyset,\{\emptyset\} \Rightarrow False$
   
    \subsection{TODO: Exercise 24}
    Determine wether each of these sets is the power set of a set, where \emph{a} and \emph{b} are distinct elements.\\
        a) $\emptyset$\\
        b) $\{\emptyset,\{a\}\}$\\
        c) $\{\emptyset, \{a\},\{\emptyset,a\}\}$\\
        d) $\{\emptyset,\{a\},\{b\},\{a,b\}\}$

    \pagebreak     
    \section{Chapter 2.2 - Set Operations}
    \subsection{TODO: Exercise 18c}
    Let A, B and C be sets. Show that:
    $A \cap B) \subseteq (A \cup B \cup C)$
    \begin{equation}
        \begin{split}
            A \cap B) \subseteq (A \cup B \cup C)
        \end{split}
    \end{equation}
    \subsection{TODO: Exercise 18d}
    Let A, B and C be sets. Show that:
    $(A - B) - C \subseteq A - C$
    \begin{equation}
        \begin{split}
            (A - B) - C \subseteq A - C
        \end{split}
    \end{equation}
    \subsection{TODO: Exercise 46}
    Show that if A, B, and C are sets, then:
    \footnote{This is a special case of the inclusion-exclusion principle, which will be studied in Chapter 8.}
    \begin{equation}
        |A \cup B \cup C| = 
        |A| + |B| + |C| - |A \cap B| - |A \cap C| - |B \cap C| + |A \cap B \cap C|
    \end{equation}

    % Learn more: https://www.sharelatex.com/learn/Aligning_equations_with_amsmath
    \begin{equation}
        \begin{split}
            |A \cup B \cup C| =& 
              |A| + |B| + |C| - |A \cap B| 
            - |A \cap C| - |B \cap C| + |A \cap B \cap C|       
            \\
            |A \cup B \cup C| =& 
            |A \cup B \cup C|
        \end{split}
    \end{equation}

    \pagebreak     
    \section{Chapter 2.3 - Functions}
    \subsection{TODO: Exercise 12c}
    Determine whether each of these functions from $Z$ to $Z$ is one-to-one.
        \begin{equation}
            f(n) = n^3
        \end{equation}

    \subsection{TODO: Exercise 38}
    Let $f(x) = ax+b$ and $g(x)= cx+d$, where a,b,c, and d are constants. Determine necessary and sufficient conditions on the constants a,b,c, and d so that $f \cdot g = g \cdot f$. 
    \subsection{TODO: Exercise 42}
    Let $f$ be the function from $R$ to $R$ defined by $f(x)=x^2$. Find:\\
    a) $f^{-1}(\{1\})$\\
    b) $f^{-1}(\{ x | 0 < x < 1\})$\\
    c) $f^{-1}(\{ x | x > 4\})$
    
    \pagebreak     
    \section{Chapter 2.4 - Sequences and Summations}
    \subsection{TODO: Exercise 12c}
    Show that the sequence {$a_n$} is a solution of the recurrence relation $a_n=-3a_{n-1}+4a_{n-2}$ if $a_n = (-4)^n$
    \begin{equation}
        \begin{split}
            a_n = -3a_{n-1}+4a_{n-2} = (-4)^n
        \end{split}
    \end{equation}

    \subsection{Exercise 33d}
    Compute the double sum
    \begin{equation}
        \begin{split}
            \sum_{i=0}^{2}\sum_{j=1}^{3}ij &= (0*1+0*2+0*3)+(1*1+1*2+1*3)+(2*1+2*2+2*3) \\
            &= (0+0+0) + (1+2+3) + (2+4+6)\\ 
            &= 0 + 6 + 12\\ 
            &= 18 
        \end{split}
    \end{equation}

    \pagebreak     
    \section{Chapter 2.5 - Cardinality of Sets}
    \subsection{TODO: Exercise 16}
    \emph{Exercise}: 
    Show that a subset of a countable set is also countable.\\
    \emph{Answer}:
    A set is countable if it is finite or is the same size as $N$.
    To show that $A$ is countable, it is sufficient to show that there is an injection
    from $A$ to $N$.

    \end{document}